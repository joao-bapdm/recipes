% Document laying out proposed scrum/kaban work methodology

\documentclass{article}

\title{Proposed workflow}
\author{Following SCRUM, trying to use KANBAN as well}
\date{}

\begin{document}

\maketitle

\section{Define products (maybe projects?)}
The stakeholders define what they want, and why do they want it.
From that, the owner should define a product (or rather a project?),
being able to prioritize demand according to importance and taking
dependency into account (relevant here?). Do a requirement list.

\section{Set sprints (or boards more appropriately)}
Instead of sprints defined in terms of time, do boards defined with
objectives. From the whole backlog list (which exists), select those
which will define the \textit{sprint}. Defining deliverables is
good as well.

\textbf{IMPORTANT}: Try to follow the plan.

\section{Review}

\subsection{Daily}
You could ask yourself:
\begin{itemize}
\item What did I do today?
\item What will I do tomorrow?
\item What are my impedments?
\end{itemize}

\subsection{Wraping up}
At the end of a \textit{sprint}, you should as yourself:
\begin{itemize}
\item From the objectives, what was achieved?
\item What wasn't achieved?
\end{itemize}
More generaly, ask yourself why did some stuff worked, and why
some stuff didn't. Ideally, this should hint at areas where improvements
could be made.

\end{document}
